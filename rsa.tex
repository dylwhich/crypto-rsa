\documentclass[10pt]{article}

\bibliographystyle{ieeetr}

\usepackage{multirow}
\usepackage{titlesec}
\usepackage{multicol}
\usepackage{tabularx}
\usepackage{longtable}
\usepackage{hyperref}
\usepackage{verbatim}
\usepackage{listings}

\lstset{
  language=python,
  basicstyle=\ttfamily,
  breaklines=true,
  columns=fullflexible
  }

\begin{document}

\begin{titlepage}
\begin{center}
\textsc{\LARGE CMSC 443}\\[.5cm]
\textsc{\Large Dylan Whichard}\\[.5cm]
\textsc{\Large Dr. Zieglar}\\[.4cm]

\rule{\linewidth}{.05mm} \\[0.4cm]

{\huge \bfseries Hard RSA Cipher}\\[.4cm]
\rule{\linewidth}{.05mm} \\[0.4cm]

\vfill

{\large \today}\\
{\large Solved: May 12, 2015 4:45pm}

\end{center}
\end{titlepage}

\section{Ciphertext}
\paragraph{}
\small{\texttt{\verbatiminput{hard_ciphertext.txt}}}

\section{Key}
\paragraph{}
\[a = │5684203361168418697394927389049400401188490076253517520708545189449027540530865526606666195\]
\[b = 10435733847851623629448058676683403047996735446407327640802538255018657950473929330040336155\]
\[p = 2577030773300010453665537724988958390877393737\]
\[q = 8904933665769254141002975302442633544151412757\]

Where \(a\) is the private exponent, \(b\) is the public exponent, and \(p\) and \(q\) are the prime factors of \(n\).

\section{Plaintext}
\paragraph{}
\texttt{\input{hard_plaintext.txt}}

\section{Methodology}
\paragraph{}
My approach to breaking this cipher was to use a program called
\texttt{GGNFS}\cite{ggnfs-source}\cite{ggnfs-guide} along with
\texttt{msieve}\cite{msieve} via a wrapper script\cite{factmsieve}
(with slight modifications required make it run), to factor the public key. It
was then trivial to write my own script\cite{rsa} which used Python's
built-in support for big-number arithmetic and fast modular
exponentiation to perform the decryption action with these derived
values. In my solution, I also used an existing piece of
code\cite{euclid} to perform the extended Euclidean algorithm for
modular inversion.

\section{Time Spent}
\paragraph{}
I spent approximately 8 hours working on the solution, and
approximately 30 minutes on this writeup.

\pagebreak
\begin{thebibliography}{1}
\bibitem{ggnfs-source} \url{http://sourceforge.net/projects/ggnfs/}
\bibitem{ggnfs-guide} \url{http://gilchrist.ca/jeff/factoring/nfs_beginners_guide.html}
\bibitem{nsieve} \url{http://sourceforge.net/projects/msieve/}
\bibitem{factmsieve} \url{https://github.com/GDSSecurity/cloud-and-control/blob/master/scripts/gengnfsjob-testharness/factmsieve.74.py}
\bibitem{rsa} See below.
\bibitem{euclid} \url{http://rosettacode.org/mw/index.php?title=Modular_inverse&oldid=196584#Python}
\end{thebibliography}

\pagebreak
\center{\Large{\textsc{rsa.py}}}\\[.15cm]
\hrule
\lstinputlisting{rsa.py}

\end{document}
